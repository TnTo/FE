\documentclass[]{beamer}

\usepackage[style=authoryear]{biblatex}
\usepackage{hyperref}

\usepackage{booktabs}
\usepackage{tabularx}
\newcolumntype{Y}{>{\raggedright\arraybackslash}X}
\usepackage{amssymb}
\usepackage{amsmath}

\AtBeginEnvironment{tabularx}{\scriptsize}
\AtBeginEnvironment{tabular}{\small}
\AtBeginEnvironment{gather}{\small}

\usetheme{CambridgeUS}
\usecolortheme{beaver}

\setbeamerfont{caption}{size=\tiny}
\setbeamertemplate{caption}{\raggedright\insertcaption\par}



\title[DAS]{An AB-SFC Macroeconomic Model with an explicit distribution of income and wealth}
\author[Ciruzzi, M.]{Michele Ciruzzi}
\institute[UnInsubria]{Università degli Studi dell'Insubria}
\date{EAEPE 2023 Conference}

\begin{document}

\frame{\titlepage}

\section{Introduction}

\begin{frame}
	\frametitle{The model}
	\begin{itemize}
		\item The model aims to represent personal inequality of income and wealth distribution
		\item It is Agent-Based Stock-Flow Consistent Macroeconomic Model with two disaggregated sectors (Households, Firms)
		\item Some choices are made to test different welfare policies in a second stage
		\item European (Italian) economy is considered as real benchmark for stylized facts
	\end{itemize}
\end{frame}

\section{The matrices}
\begin{frame}
	\frametitle{The balance matrix}
	\makebox[\textwidth][c]{
		\begin{tabular}{l|cccccc|l}
			\toprule
			             & $\mathcal{H}$             & $\mathcal{F}_{\mathbf{C}}$                              & $\mathcal{F}_{\mathbf{K}}$                              & $\mathcal{B}$             & $\mathcal{G}$             & $\mathcal{C}$             & Tot.                         \\
			\midrule
			$\mathbf{D}$ & $+\mathbf{D}_\mathcal{H}$ & $+\mathbf{D}_{\mathcal{F}_{\mathbf{C}}}$                & $+\mathbf{D}_{\mathcal{F}_{\mathbf{K}}}$                & $-\mathbf{D}$             &                           &                           & 0                            \\
			$\mathbf{S}$ & $+\mathbf{S}$             &                                                         &                                                         & $-\mathbf{S}$             &                           &                           & 0                            \\
			$\mathbf{L}$ &                           & $-\mathbf{L}^{\mathcal{F}_{\mathbf{C}}}$                & $-\mathbf{L}^{\mathcal{F}_{\mathbf{K}}}$                & $+\mathbf{L}$             &                           &                           & 0                            \\
			$\mathbf{B}$ &                           &                                                         &                                                         & $+\mathbf{B}_\mathcal{B}$ & $-\mathbf{B}$             & $+\mathbf{B}_\mathcal{C}$ & 0                            \\
			$\mathbf{R}$ &                           &                                                         &                                                         & $+\mathbf{R}_\mathcal{B}$ & $+\mathbf{R}_\mathcal{G}$ & $-\mathbf{R}$             & 0                            \\
			$\mathbf{K}$ &                           & $+p_{\mathbf{K}} \mathbf{K}_{\mathcal{F}_{\mathbf{C}}}$ & $+p_{\mathbf{K}} \mathbf{K}_{\mathcal{F}_{\mathbf{K}}}$ &                           &                           &                           & $+p_{\mathbf{K}} \mathbf{K}$ \\
			\midrule
			Tot.         & $+V_\mathcal{H}$          & $+V_{\mathcal{F}_{\mathbf{C}}}$                         & $+V_{\mathcal{F}_{\mathbf{K}}}$                         & $+V_\mathcal{B}$          & $+V_\mathcal{G}$          & $+V_\mathcal{C}$          & $+p_{\mathbf{K}} \mathbf{K}$ \\
			\bottomrule
		\end{tabular}
	}\\
\end{frame}

\begin{frame}
	\frametitle{The transactions matrix}
	\makebox[\textwidth][c]{
		\begin{tabularx}{\textwidth}{@{} Y|cccccc|l @{}}
			\toprule
			                       & $\mathcal{H}$                            & $\mathcal{F}_{\mathbf{C}}$                              & $\mathcal{F}_{\mathbf{K}}$                                             & $\mathcal{B}$                          & $\mathcal{G}$                            & $\mathcal{C}$                          & Tot. \\
			\midrule
			Consumption            & $-p_{\mathbf{C}} \mathbf{C}_\mathcal{H}$ & $+p_{\mathbf{C}} \mathbf{C}$                            &                                                                        &                                        & $-p_{\mathbf{C}} \mathbf{C}_\mathcal{G}$ &                                        & 0    \\
			Investment             &                                          & $-p_{\mathbf{K}} \mathbf{K}_{\mathcal{F}_{\mathbf{C}}}$ & $+p_{\mathbf{K}} (\mathbf{K} - \mathbf{K}_{\mathcal{F}_{\mathbf{K}}})$ &                                        &                                          &                                        & 0    \\
			Wages                  & $+W$                                     & $-W^{\mathcal{F}_{\mathbf{C}}}$                         & $-W^{\mathcal{F}_{\mathbf{K}}}$                                        &                                        &                                          &                                        & 0    \\
			Taxes                  & $-T$                                     &                                                         &                                                                        &                                        & $+T$                                     &                                        & 0    \\
			Transfers              & $+M$                                     &                                                         &                                                                        &                                        & $-M$                                     &                                        & 0    \\
			\midrule
			$\mathcal{F}$ Profits  &                                          & $-\Pi^{\mathcal{F}_{\mathbf{C}}}_\mathcal{B}$           & $-\Pi^{\mathcal{F}_{\mathbf{K}}}_\mathcal{B}$                          & $+\Pi_\mathcal{B}$                     &                                          &                                        & 0    \\
			$\mathcal{C}$ Profits  &                                          &                                                         &                                                                        &                                        & $+\Pi^\mathcal{C}$                       & $-\Pi^\mathcal{C}$                     & 0    \\


			\midrule
			$\mathbf{S}$ Interests & $+r_{\mathbf{S}} \mathbf{S}$             &                                                         &                                                                        & $-r_{\mathbf{S}} \mathbf{S}$           &                                          &                                        & 0    \\
			$\mathbf{L}$ Interests &                                          & $-r_{\mathbf{L}} \mathbf{L}^{\mathcal{F}_{\mathbf{C}}}$ & $-r_{\mathbf{L}} \mathbf{L}^{\mathcal{F}_{\mathbf{K}}}$                & $+r_{\mathbf{L}} \mathbf{L}$           &                                          &                                        & 0    \\
			$\mathbf{B}$ Interests &                                          &                                                         &                                                                        & $+r_\mathbf{B} \mathbf{B}_\mathcal{B}$ & $-r_\mathbf{B} \mathbf{B}$               & $+r_\mathbf{B} \mathbf{B}_\mathcal{C}$ & 0    \\
			\bottomrule
		\end{tabularx}
	}\\
\end{frame}

\section{The stocks}
\begin{frame}
	\frametitle{Capital goods}
	A capital good is characterized by:
	\begin{itemize}
		\item a minimum skill level required to operate it $\sigma$
		\item the output produced by the worker operating it $\beta$
		\item constant life-time with constant depreciation
	\end{itemize}
\end{frame}

\section{The agents}
\begin{frame}
	\frametitle{Households}
	\begin{itemize}
		\item Heterogeneous in their skill level $\sigma_t = (1+\Sigma)^\delta \sigma_{t-1}$
		\item Retired at fixed age and get substituted by a single new agent which inherits the wealth
		\item Enter the simulation with an average initial skill $\mathbb{E}(\sigma_0) = 1+(\sigma^M-1)\tanh(e_0 \frac{v}{p})$
	\end{itemize}
\end{frame}

\begin{frame}
	\frametitle{Households}
	Marginal propensity to consume is assumed decreasing in wealth $\eta_t = (\frac{v_{t-1}}{p_{t-1}} + 1)^{-a}$

	\begin{gather*}
		\begin{align*}
			\eta {\Delta c}_t & = {\Delta y}_t  = {\Delta z}_t + {\Delta (r_\mathbf{S} \mathbf{s})}_t   = {\Delta z}_t + ({r_\mathbf{S}}_t + {\Delta r_\mathbf{S}}_{t})\mathbf{s}^*_t - {r_\mathbf{S}}_{t} \mathbf{s}_{t-1} \\
			                  & \approx {\Delta z}_t   + ({r_\mathbf{S}}_t + {\Delta r_\mathbf{S}}_{t})(\mathbf{s}_{t-1} + \mathbf{d}_{t-1} + {\mathbb{E}(z)}_t + {\mathbb{E}(m)}_t - \mathbf{d}^*_t - c^*_t)               \\
			                  & \qquad - {r_\mathbf{S}}_{t} \mathbf{s}_{t-1}                                                                                                                                                \\
			                  & = ({r_\mathbf{S}}_t + {\Delta r_\mathbf{S}}_{t})(\mathbf{d}_{t-1} + z_{t-1} + \phi m_{t-1} - (1+\rho_{\mathcal{H}}) c^*_t)                                                                  \\
			                  & \qquad+ {\Delta r_\mathbf{S}}_{t} \mathbf{s}_{t-1}                                                                                                                                          \\
		\end{align*}\\
		{\Delta c}_t \approx \frac{({r_\mathbf{S}}_t + {\Delta r_\mathbf{S}}_{t})(\mathbf{d}_{t-1} + z_{t-1} + \phi m_{t-1} - (1+\rho_{\mathcal{H}}) c_{t-1}) + {\Delta r_\mathbf{S}}_{t} \mathbf{s}_{t-1}}{\eta +  ({r_\mathbf{S}}_t + {\Delta r_\mathbf{S}}_{t})(1+\rho_{\mathcal{H}})}
	\end{gather*}
\end{frame}

\begin{frame}
	\frametitle{Consumption Firms}
	\vspace{-1em}
	\begin{gather*}
		\mathbf{c}^*_t = \rho_\mathbf{C}(1+g-\psi)s_{t-1} = \rho_\mathbf{C}{\mathbb{E}(s)}_t \\
		b^*_t = \frac{1}{u^*}\mathbf{c}^*_t + \gamma b_{t-1} \\
		i^*_t = (1+\psi){\langle\frac{{p_\mathbf{K}}}{\beta}\rangle}_{t-1}{\Delta b}_t \\
		w_t^* = \max(w_{t-1}, \frac{\mathbf{c}^*_t}{{\langle \beta \rangle}_{t-1}}{\langle w \rangle}_{t-1}) \\
		l_t^* = \max(\rho_\mathcal{F} w_t^* - \mathbf{d}_{t-1}, \rho_\mathcal{F} (w_t^* + i_t^*) - (\mathbf{d}_{t-1} + (1+\psi){p_\mathbf{C}}_{t-1} {\mathbb{E}(s)}_t), 0) \\
		\mu_t = \mu_{t-1}(1 + \Theta \frac{s_{t-1}-{\mathbb{E}(s)}_{t-1}}{{\mathbb{E}(s)}_{t-1}}) \\
		{p_\mathbf{C}}_t = (1+\tau_\mathbf{C})(1+\mu_t)\frac{w_t}{\mathbf{c}_t} \\
		\pi_t = r_\Pi ({p_\mathbf{C}}_t s_t - w_t)
	\end{gather*}
\end{frame}

\begin{frame}
	\frametitle{Capital Firms}
	\begin{gather*}
		\mathbf{k}_t^* = \rho_\mathbf{K}(1+g-\psi)s_{t-1} + \frac{{\Delta b}_t}{\langle \beta \rangle} - \hat{\mathbf{k}}_{t-1} \\
		b^*_t = \rho_\mathbf{K}\frac{(1+g-\psi)}{u^*}s_{t-1} + \gamma b_{t-1} \\
		i^*_t = 0, \qquad l_t^* = \max(\rho_\mathcal{F} w_t^* - \mathbf{d}_{t-1}, 0) \\
		q_t^* = \lfloor q_{t-1} (1+\frac{\rho_Q}{s_{t-1}}\frac{\pi_{t-1}}{{\langle w_Q \rangle}_{t-1}}) \rfloor
	\end{gather*}
	Innovate with probability $\theta = e^{-\zeta q}$ with output $\Delta \beta_\mathbf{C} = \text{Beta}(1, b_0)$ $\Delta \sigma = (\Delta \beta_\mathbf{C} - b_1 \text{Beta}(1, b_2))$

\end{frame}

\begin{frame}
	\frametitle{Bank}
	Represent the aggregate financial sector. No enforceable liquidity ratio. Target capital ratio $\Gamma = \frac{V}{L}$.

	\begin{gather*}
		{r_\mathbf{S}}_t = (1 - \tau_\mathbf{S}) ({r_\mathbf{B}}_t + \lambda(\Gamma - \Gamma^*))\\
		{l^f}^*_t = \min (\nu_0 (p_\mathbf{K}{\mathbf{k}_f})_{t-1} - \mathbf{l}^f_{t-1}, \max((\nu_1 N_F)^{-1} \mathbf{L}_{t-1} (\frac{\Gamma_{t-1}}{\Gamma^*}-1),0)) \\
		{r_\mathbf{L}}^f_t = r_\mathbf{B} + \nu_2 (\Gamma^* - \Gamma_{t-1}) + \nu_3 (\frac{\mathbf{l}^f_{t-1}}{{v_f}_{t-1}}) - \nu_4 (\frac{{\pi_f}_{t-1}}{{\mathbf{l}^f}_{t-1}})
	\end{gather*}
\end{frame}

\begin{frame}
	\frametitle{Government}
	Maastricht-like setting.

	\begin{gather*}
		m_{t} = \phi \max(w_{t-1}, m_{t-1}), \qquad {\mathbf{c}^\mathcal{G}_h}_t = ((1 - \varepsilon_0) + \varepsilon_0 e^{-\varepsilon_1\frac{{v_h}_{t-1}}{p}})\Xi_t\\
		(1+g) \delta^* Y_{t-1}  =  {\mathbb{E}(G)}_t + {r_\mathbf{B}}_t {\mathbf{B}_\mathcal{B}}_{t-1} (1 + \frac{(1+g) \delta^* Y_{t-1}}{{\mathbf{B}^\mathcal{G}}_{t-1}}) - (1+g+\psi) T_{t-1} \\
		\mathbb{E}(G)  = (1+\psi+g) \langle T \rangle + (1+g)(1-{r_\mathbf{B}}_t \langle \frac{{\mathbf{B}_\mathcal{B}}}{{\mathbf{B}^\mathcal{G}}} \rangle)\delta^* Y_{t-1} - {r_\mathbf{B}}_t {\mathbf{B}_\mathcal{B}}_{t-1} \\
		\mathbb{E}(G)_t = \mathbb{E}(M)_t + \mathbb{E}(C^\mathcal{G})_t \approx M_{t-1} + (1+\psi) p \frac{{\mathbf{C}^\mathcal{G}}_{t-1}}{\Xi_{t-1}} \Xi_t \\
		\Xi_t          = \frac{\Xi_{t-1}}{\langle {\mathbf{C}^\mathcal{G}} \rangle} \frac{\mathbb{E}(G) - \langle M \rangle}{(1+\psi)p}
	\end{gather*}
\end{frame}

\begin{frame}
	\frametitle{Central Bank}
	$${r_\mathbf{B}} = \psi + \alpha_1 (\psi - \psi^*) + \alpha_2 (u - u^*) - \alpha_3 (\omega - \omega^*)$$
\end{frame}

\section{Dynamics}
\begin{frame}
	\frametitle{Time}
	To smooth the dynamic of the model, two different times are overlapped
	\begin{itemize}
		\item a slow-time (the quarter) which is used to update institutional decision (CB rate, government targets, aggregate statistics update)
		\item a fast-time (the month) which is used for consumption and production
	\end{itemize}
\end{frame}

\begin{frame}
	\frametitle{Good Market}
	For each transaction, an agent buys $\chi^{-1}$ of the desired quantity $\chi$ times looking at $\chi$ sellers.

	The markets in the model are
	\begin{itemize}
		\item Consumption Goods market: Consumption Firms sell Consumptions Goods to Households and the Government
		\item Capital Goods market: Capital Firms sell Capital Goods to Consumption Firms
	\end{itemize}

	Every other transaction is settled at will, given the constraints.
\end{frame}

\begin{frame}
	\frametitle{Labour Market}
	Households are employed by a Firm until they are fired (when $w_t - p_t s_t < 0$), they chose to exit the job market or they accept an offer from another Firm.

	For each vacancy (i.e. unmatched capital good required for production target or research position) a Firm sees $\chi_\mathcal{H}$ workers which have the required skills and earn less than the average salary in the model for that skill level.

	The Firm employs the one with the higher skills offering the average salary in the model for the required skill level.
\end{frame}

\section{Further steps}
\begin{frame}
	\frametitle{Implementation}
	The planned implementation will rely on a stateful description in which it is possible to reconstruct most of the dynamics of the model.

	Particularly the state of the model at the each of each time-step will be stored.

	The model is written in Julia (\url{https://github.com/TnTo/FE/}).

	This is helping prototyping since the quantities to be measured and recorded are not required to be defined a priori.
\end{frame}

\begin{frame}
	\frametitle{Calibration}
	The calibration will aim not to fit real-world time series, but to match stylized facts.

	The target facts will be listed as expected ranges of macroeconomic variables or moments of distributions, taking as reference the Western-European economies.

	A Monte-Carlo search on the parameter space will be used to set the parameters that have no clear economic meaning to match the stylized facts.
\end{frame}

\begin{frame}
	\frametitle{Policies experiments}
	The final goal of the work is to compare the macroeconomic effects of different welfare paradigms.
	\begin{itemize}
		\item A Keynesian government with full-employment or poverty-reduction goals
		\item Minimum wage
		\item Universal Basic Income
		\item Universal Basic Dividend
		\item Universal Basic Service (more significant with two-goods/foundational perspective)
		\item Job Guarantee scheme (with differentiated firm-property and public ownership)
		\item Cooperativism (with differentiated firm-property and public ownership)
	\end{itemize}
\end{frame}

\begin{frame}
	\frametitle{What is missing}
	\begin{itemize}
		\item A gender perspective: the care and domestic work and the education paths
		\item A spatial perspective: commuting and housing -- a mayor factor of financial stress for the many --
		\item A foundational perspective: differentiated goods and the public infrastructure
		\item A capital perspective: competitive models of firm ownership and profit distribution
		\item A development perspective: wage suppression and export-led growth in a multi-country model
	\end{itemize}
\end{frame}

\begin{frame}
	\centering
	Thanks for your attention!

	Any question?
\end{frame}

\end{document}
