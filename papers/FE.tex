% !TeX spellcheck = en_GB
\documentclass[a4paper, headings=standardclasses]{scrartcl}

\usepackage[margin=2.5cm]{geometry}
\usepackage{etoolbox}
\usepackage{authblk}
\renewcommand{\Affilfont}{\small}
\newcommand\blfootnote[1]{%
  \begingroup
  \renewcommand\thefootnote{}\footnote{#1}%
  \addtocounter{footnote}{-1}%
  \endgroup
}
\usepackage[style=authoryear, backend=biber, sorting=nyt, useprefix=true]{biblatex}
\usepackage[autostyle=false, style=english]{csquotes}
\MakeOuterQuote{"}
\usepackage[british]{babel}
\usepackage[modulo]{lineno}
\linenumbers
\usepackage{framed}
\usepackage[hidelinks]{hyperref}
\usepackage{booktabs}
\usepackage{tabularx}
\newcolumntype{Y}{>{\raggedright\arraybackslash}X}
\usepackage{amssymb}
\usepackage{color}
\usepackage{soul}
\usepackage{enumitem}
\usepackage{cleveref}

\addbibresource{FE.bib}

\newenvironment{enh}[1][]{\begin{framed}\noindent\textbf{Enhancement: #1}\par}{\end{framed}}

\newcommand{\todo}[1]{\par \textbf{ToDo:} #1}

\newlist{steps}{enumerate}{1}
\setlist[steps]{noitemsep,label=(\arabic*)}

\crefname{stepsi}{step}{steps}
\Crefname{stepsi}{Step}{Steps}


%opening
\title{A Hybrid AB-SFC Macroeconomic Model with explicit distribution of income and wealth \let\thefootnote\relax\footnotetext{
	An updated version of this paper and all the source code and the instructions required to replicate the paper are available at \url{https://github.com/TnTo/FE/}

	\hl{Highlighted} parts of the text indicate substantial choices to be taken.
  }}
\subtitle{Working Notes}
\author{Michele Ciruzzi\thanks{mciruzzi@uninsubria.it - \url{https://orcid.org/0000-0003-1485-1204}}}

\begin{document}

\maketitle

%\begin{abstract}
%\end{abstract}

\section{Introduction}
\subsection{Aims}
The long-term goal of this model is to highlight the macroeconomic and distributional effects of some welfare policies.
The focus is put in particular on some (recent) policies yet unapplied in the real world as Universal Basic Income, Job Guarantee schemes, or the presence of only cooperative firms.

\begin{enh}[Foundational Economy -- 1]
	To better characterize the differences in behaviour among low- and high-income households it is possible to refer to the theoretical framework provided by Foundational Economy \parencite{arcidiacono2018}. It suggests that a significant part of the economic activities are instrumental not to the extraction of rents from capital, but to address essential needs and to build up shared infrastructures
	\begin{quote}
		"It argues that the well-being of Europe's citizens depends less on individual consumption and more on their social consumption of essential goods and services – from water and retail banking, to schools and care homes – in what we call the foundational economy. Individual consumption depends on market income, while foundational consumption depends on infrastructure and delivery systems of networks and branches, which are neither created nor renewed automatically, even as incomes increase. The distinctive, primary role of public policy should therefore be to secure the supply of basic services for all citizens. If the aim is citizen well-being and flourishing for the many not the few, then European politics at regional, national and EU level needs to be refocused on foundational consumption and securing universal minimum access and quality." \parencite{arcidiacono2018}
	\end{quote}
\end{enh}

\begin{enh}[Welfare policies]
	The first version of the model will be as simple as possible to create a robust baseline.
	Subsequent iterations of the model will explore different welfare policies and how to model them.
\end{enh}

\section{General Hypothesis}
\subsection{Time}
The timescale of the first version of the model should be relevant to calibrate the model on real data.

It is possible that an adaptive approach for the agents' behaviour works better using a higher frequency model that covers a shorter timespan (e.g. one month per tick, 15 years length, 180 time steps in total), because of the smaller variations expected at each tick.

Moreover, in a future version of the model, the simulation's timespan has to be long enough to observe the effects of introducing a policy. But, at the same time, it is unreasonable to keep the simulation running over 5-10 years after the policy's introduction because, in any real-world context, a government can tune or revert the policy afterwards.

\subsection{Close Economy}
The assumption of a close economy strongly reduces the complexity of the model but prevents observing some economic phenomena like export-led growth (such as for Italy or Germany) or the offshoring of labour-intensive productions.
Nevertheless, this is a common hypothesis which is used also in this model.

\begin{enh}[Multi-Country Model]
	A compromise for future development is to model in an AB-SFC setting the main economy of the model while keeping aggregated (SFC only) the other economies.
\end{enh}

\subsection{Sectors}
The model includes the core sector of most SFC models \parencite{nikiforos2017}. Of those, three (Banks ($B$), Government ($G$) and Central Bank ($C$)) are represented by a single agent because unique in the model or described as an aggregate sector, while the remaining two (Households ($H$) and Firms ($F$)) are disaggregated and constitute the Agent-Based part of the model.

Firms are considered as different sectors in the model matrices depending on the goods produced.

\subsection{Real Assets}
The model comprises two kinds of real assets: Capital Goods ($K$) and Consumption Goods ($G$).
Only the Capital Goods are durable.

\begin{enh}[Foundational Economy -- 2]
	To model the idea of a Foundational Economy the single representative Consumption Good can be splitted in a representative Foundational Consuption Good and another kind of non-Foundational Consumption Goods.
	The intuition which can be followed is that each Household aims to consume a certain amount of Foundational Goods before to start consuming the other Goods. Additionally the non-monetary transfers by the Government should comprise only Foundational Goods.
\end{enh}

\subsection{Financial Assets}
The model includes five different financial assets.
Bank Deposits ($D$) of Households and Firms, which are not interest-bearing.
Loans ($L$) issued by the Banks to Firms, which interest rate is Firm-specific and fixed by the Bank.
Bank Bonds ($S$, like shares) held by Households, which interest rate is fixed each period by the Bank.
Banks Reserves and Government Account at the Central Bank ($R$), which are not interest-bearing.
Government Bonds ($T$, like treasure bonds) hold by Bank and Central Bank, and which interest rate is fixed by the Central Bank.

\section{Matrices}
\subsection{Balance Sheet Matrix}
\makebox[\textwidth][c]{
	\begin{tabular}{l|ccccccc|l}
		\toprule
		     & $H$    & $F_C$        & $F_K$        & $B$    & $G$    & $C$    & Tot.   \\
		\midrule
		$D$  & $+D_H$ & $+D_{F_C}$   & $+D_{F_K}$   & $-D$   &        &        & 0      \\
		$S$  & $+S_H$ &              &              & $-S$   &        &        & 0      \\
		$L$  &        & $-L^{F_C}$   & $-L^{F_K}$   & $+L$   &        &        & 0      \\
		$T$  &        &              &              & $+T_B$ & $-T$   & $+T_C$ & 0      \\
		$R$  &        &              &              & $+R_B$ & $+R_G$ & $-R$   & 0      \\
		$K$  &        & $+p K_{F_C}$ & $+p K_{F_K}$ &        &        &        & $+p K$ \\
		\midrule
		Tot. & $+V_H$ & $+V_{F_C}$   & $+V_{F_K}$   & $+V_B$ & $+V_G$ & $+V_C$ & $+p K$ \\
		\bottomrule
	\end{tabular}
}\\ \\
$V$ is the Net Worth of the sector.

\subsection{Transactions Matrix}
\makebox[\textwidth][c]{
	\begin{tabularx}{\textwidth}{@{} Y|ccccccc|l @{}}
		\toprule
		              & $H$      & $F_C$        & $F_K$              & $B$      & $G$    & $C$      & Tot.     \\
		\midrule
		Consumption   & $-p G_H$ & $+p G$       &                    &          &        & $-p G_G$ &      & 0 \\
		Investment    &          & $-p K_{F_C}$ & $+p (K - K_{F_K})$ &          &        &          & 0        \\
		Wages         & $+W$     & $-W^{F_C}$   & $-W^{F_K}$         &          &        &          & 0        \\
		Taxes         & $-T$     &              &                    &          & $+T$   &          & 0        \\
		Transfers     & $+M$     &              &                    &          & $-M$   &          & 0        \\
		\midrule
		$F$ Profits   &          & $-\Pi^{F_C}$ & $-\Pi^{F_K}$       & $+\Pi$   &        &          & 0        \\
		$C$ Profits   &          &              &                    &          & $+\Pi$ & $-\Pi$   & 0        \\


		\midrule
		$S$ Interests & $+r S$   &              &                    & $-r S$   &        &          & 0        \\
		$L$ Interests &          & $-r L^{F_C}$ & $-r L^{F_K}$       & $+r L$   &        &          & 0        \\
		$T$ Interests &          &              &                    & $+r T_B$ & $-r T$ & $+r T_C$ & 0        \\
		\bottomrule
	\end{tabularx}
}\\

\section{Sectors}
\subsection{Households}
In this model, the core agents (consumer, worker, capitalist) represent a household rather than a single individual. This is a very common approximation in economics and I think it is reasonable as long as we are not going into modelling education paths and care work, where the gender asymmetries become very relevant.

Each agent is characterized by an education level assigned when it enters the simulation replacing a retired agent inheriting their wealth, and gains experience when working in the same sector (Capital/Essential/Other) without an employment gap.
The education level is assigned with a probability related to the inherited wealth and provides the starting skill level.
Skills $s$ evolve like in \textcite{dosi2018}, which means $s_t = (1+\phi)^\delta s_{t-1}$ where $\delta=1$ if the household is employed in the same sector of the previous time step, $\delta=0$ if the household is still employed but in a different sector, $\delta=-1$ if the household is unemployed.

\begin{enh}[Training]
	Two factors in the development of the skills can be introduced.

	One on the welfare policies side is the possibility for the government to organize training programs for the unemployed to prevent the loss of skills on even increase them.

	The second one relates to the actual job done: it is reasonable to assume that it is easier to learn new skills if the skills required for the job are closer to the skill level, while demoted and overqualified workers have lower chances to learn new skills. This can be included in the model making $\phi \propto (s-\sigma)^{-1}$, where $\sigma$ is the minimum skill level required to operate the machinery assigned to the worker in the time step.
\end{enh}

Households face two choices: if work and which proportion of their income they should consume.

Households flows' balance is $I = W + M + rS = C + T + \Delta S + \Delta D$.
I assume, as a heuristic, that Taxes ($T$) and Transfers ($M$) can be approximated as constant from the previous period.
Additionally, I assume that desired Deposits ($D$) at the end of the period are a fraction of the desired consumption ($C = \langle p_G \rangle G$) as insurance against unexpected increases in prices \hl{or unemployment}\footnote{It is possible to assume that deposits are used only in case of increase in prices, which allows setting $\rho$ smaller than one, assuming that in case of unexpected unemployment a mix of public subsidies and cashing out from Bank Bonds, without getting interests paid, is pursued. Otherwise, if deposits are used also as insurance against unemployment $\rho$ as to be greater than 1.} ($D = \rho~C$, $\rho > 0$).
Subtracting two consecutive periods and ignoring second-order differences, the in-flow income becomes $\Delta I = \Delta W + \Delta r S + r \Delta S$, and calling $\eta$ the marginal propensity to consume it becomes $\eta \Delta C \approx \Delta W + \Delta r S + r \Delta S$.
Fixed the income level $I$ we can write $0 = \Delta C + \Delta D + \Delta S = (1+\rho) \Delta C + \Delta S$ to express the choice the household faces between consumption and saving.
Putting these together we find $\eta \Delta C \approx \Delta W + \Delta r S - r (1+\rho) \Delta C$ and so $\Delta C \approx \frac{\Delta W + \Delta r S}{\eta + r (1+\rho)}$, which provides an adaptive rule for monetary consumption.

To translate this in material terms I introduce the material consumption of consumption goods $G$. We have $\Delta C = \Delta (p G) \approx \Delta(p) G + p \Delta G \approx p (\pi G + \Delta G)$. We can finally rewrite $\Delta G = \frac{\Delta W + \Delta r S}{p (\eta + r (1+\rho))} - \pi G$

\hl{$\Delta W$ can be approximated by $w W_{t-1}$ where $w$ is the average rate at which wages for the given skill level are increased in the last year}\footnote{Or 0 assuming that salaries increase only changing employer which is an unexpected event.}. $\Delta r$ is communicated by the Bank before the agent has to choose between consumption and saving. $\eta$ is calibrated from empirical data as an exponential or Pareto distribution as $\eta(\frac{S}{p},\frac{W}{p})$, where the price is used to get a-dimensional values \parencite{fisher2020,carroll2017}.

From this relation, we can model the two choices.

First, a household exits from the labour market if the loss of the wage can anyway grant an increase in consumption, i.e. $\Delta r S - p \pi (\eta + r (1+\rho)) G > W$. Similarly, it re-enters the labour market if the expected salary (i.e. the average salary given the skill level) prevents a loss of consumption, i.e. if $\Delta r S - p \pi (\eta + r (1+\rho)) G < 0$ and $W + \Delta r S - p \pi (\eta + r (1+\rho)) G > 0$.

Second, each household sets the desired total consumption as $\mathbb{E}(G) = G + \Delta G$ (using $G$ from the previous period). From which follows $\mathbb{E}(C) = p (1+\pi) \mathbb{E}(G)$, $\mathbb{E}(D) = \rho p (1+\pi) \mathbb{E}(G)$ and $\Delta S = D + (1+w) W + M - (1+\rho) \mathbb{E}(C)$, where $D$ is the value of the deposit at the end of the previous time step.

\begin{enh}[Gender, Care work and Feminist Economics]
	Approximating individuals as household invisibilizes gender differences and the (hidden) work made mostly by women inside the family (childcare, elder care, housekeeping, ...).
	Gender is an important factor in creating inequalities: for example, unemployment and wages show a strong gender effect (which in both cases penalizes women).

	Adding a gender perspective will be an improvement in the model (with relevant policy implications) and will require explicitly modelling education and childcare (which in this first draft is only sketched), the complete life cycle of an agent (here reduced to the working age) and family choices (marriage, pregnancy, ...).
\end{enh}

\subsection{Firms}
Firms are characterized by their position in the supply chain (either Capital or Consumption).

Loans are taken as required. In case of rationed credit, Firms first pay salaries of workers, then pay salaries of researchees, then acquire new machineries and finally repay the Bank of the Loans contracted.

\subsubsection{Consumption Firms}
Noted $S$ the quantity of goods solds in the previous period, each Firm sets the desired production as $\mathbb{E}(G) = k(1+g+\pi)S$, where $g$ is the GDP growth rate.

The maximum output of the available machinery is defined as $B = \sum_{k \in K} \beta_k$. Each Firm set a desired potential output $\mathbb{E}(B) = \frac{1}{u^*}\mathbb{E}(G) + \delta B$, where $\delta$ is the depreciation rate of capital and $u^*$ is the target capacity utilization rate. From this follows that $\Delta B = \max(\frac{\mathbb{E}(G)}{u^*} - (1 - \delta) B, 0)$ and the expected investment in monetary units is $\mathbb{E}(I) = p_K\frac{\Delta B}{\langle \beta \rangle}$.

The expected expense for salaries is $\mathbb{E}(W) = \frac{\mathbb{E}(G)}{\langle \beta \rangle}\langle w \rangle$, where $w$ are the individual salaries paid.

The markup is set equal to $\mu = \mu_{t-1}(1 + \theta \frac{S_{t-1}}{G})$ after that the production happened.

Called $U$ the balance of flows in the period before distributing profits and $D$ the amount of deposits in the same moment, each firm distributes profits $\Pi = k U \frac{D}{C}$, where $C = I + W$ in the period.


\subsubsection{Capital Firms}
Capital Firms produce their own machiney, creating a sistem of two simultaneous equations to derminate the desired output.
We avoid the problem approximating $\mathbb{E}(B) = \frac{k(1+g-\pi)}{u^*}S + \delta B$, and $\mathbb{E}(K) = k(1+g-\pi)S + \Delta B - K$, where $K$ are the unsold capital goods from the previous period. The machinery produced for own use are kept separated from those to be sold.

Additionally, Capital Firms perform research to improve the machinery they sell. They aim to employ a number of researchers $\mathbb{E}(R) = R(1+\frac{k}{S}\frac{\Pi}{\langle w_R \rangle})$ where $\langle w_R \rangle$ is the average salaries of the employees with $\sigma \ge \sigma^*$.

Finally we note that $I = 0$ and so $C=W$.

\begin{enh}[Firms' governance]

\end{enh}

\begin{enh}[Public Firms]
\end{enh}


\subsection{Bank}
Bank agent represents the aggregate banking sector.

Bank is required to maintain both a liquidity ($\Lambda = \frac{R}{D}$) and a capital ratio ($\Gamma = \frac{V}{L}$).

Liquidity is obtained, in case of necessity, by selling Government's Bonds to the Central Bank.

Bank fixes the interest rates on Bank Bonds as $r_S = (1 - \tau_S) (i + \lambda(\Gamma - \Gamma^*))$, where $i$ is the Central Bank interest rate and $\tau_S$ is the constant tax rate on financial income. In this model the Bank does not distribute profits and can access all the needed liquidity from the Central Bank, making the liquidity requirement a tautology and needed a way to avoid excessive capitalization. So, it is not possible to have $r_S(\Lambda)$ (because it will be constant) and to make $r_S(\Gamma)$ approximate a profit distribution, without in a bond-like market (rather than a stock-like ownership model).

\begin{enh}[Competitive credit market]
	There are difficulties in setting the Bank Bonds interest rate because there is no competition in the credit market for Household savings, and Central Bank provides free from interest liquidity.
	The first possibility is to disaggregate the sector and to make interest rates on Bank Bonds a tool of competition among banks.
	The other is to allow Central Bank to ration the access to liquidity, charging an interest rate on Advances (adding a financial stock in the model).
\end{enh}

It also chooses when granting loans to Firms (based on the balance sheet of the applicant) and fixes a different interest rate for each loan. The duration of Loans is fixed.
Bank is willing to provide loans to a firm $F$ up to $\hat{l} = \min (\chi_0 (D^F+pK^F) - L^F, \max((\chi_1 N_F)^{-1} L (\frac{\Gamma}{\Gamma^*}-1),0))$, at a firm-specific interest rate $r_L^F = i + \gamma_1 (\Gamma - \Gamma^*) + \gamma_2 (\frac{L^F}{V^F}) - \gamma_3 (\frac{\Pi^F}{L^F})$.
These relations account for the fulfilment of the capital requirement for the Bank and the presence of sufficient collaterals on the Firm side.

\subsection{Government}
Government fixes the public expenditure. It additionally collects taxes and pays unemployment benefits. It determines the amount to be transferred to Households (both as monetary and non-monetary, as Essential Goods).

When liquidity is needed, Government emits Bonds and sells them at will to the Central Bank.

As an approximation, fiscal policy is kept constant during the simulation and taxes are collected only from the Household sector during the transactions. Particularly, the model includes four taxes: a VAT on the purchase of consumption goods ($\tau_G$); a flat financial income tax on distributed interests on Bank Bonds ($\tau_S$); a progressive income tax computed as $T_W(W) = W \max(\tau_M \tanh(\tau_F (\frac{W}{p} - \tau_T)),0)$; an inheritance tax on wealth with constant rate ($\tau_I$).

Fixed unemployment benefit $U_{t+1} = \gamma \max{W, U}$ is paid to those who have not exited the job market and are not employed.

Each period government buys Consuption goods which distribute to the Households. Particularly each Household receives an amount of Consumption goods equal to $G^G = ((1 - q_0) + q_0 e^{-q_1\frac{V}{p}})\hat{G}$.

We define $g$ as the growth rate of the GDP (measured as Government and Households consumptions, including VAT, plus the variation in Capital stock, noted as $Y$) in the previous periods, $M = \sum_H U$ and $G = \sum_H G^G$.

Assuming a Maastricht-like scenario in which Government has a target deficit $\delta$ to achieve, the expected balance of the next period can be written as $\delta Y (1+g) = (1+\pi) (\mathbb{E}(G) + M) + i T_B(1 + \frac{1}{T^G} \delta Y) - (1+g)(1+\pi) T$ where it is assumed that expenses for unemployment benefits are approximately constant, $T$ are the taxes raised, $T_B$ the stock of Government bonds held by the Bank and $T^G$ is the total stock of Government Bonds.

Then, $\mathbb{E}(G) = (1+g) T + \frac{1+g}{1+\pi}(1-i\frac{T_B}{T^G})\delta Y - \frac{i}{1+\pi}T_B - M$, from which it is possible to write an adaptive rule for Government demand which aims to match the target deficit as $\Delta \hat{G} = q_2 \frac{\mathbb{E}(G)-G}{p N^H}$, where $N^H$ is the number of Households agents.


\subsection{Central Bank}
In the model the role of the Central Bank is to fix the Government's Bonds interest rate according to a Taylor rule $i = \pi + \alpha_1 (\pi - \pi^*) + \alpha_2 (c - c^*) - \alpha_3 (u - u^*)$, where $\pi$ is the inflation rate, $c$ is the capacity utilization measured as the fraction of capital goods used in production, $u$ is the unemployment rate computed among those who have not voluntarily exited the job market, and starred variables are the targets.

Additionally, it passively buys and sells Government Bonds on request to the Bank and the Government. Reserves do not grant interests.

In other words, the Central Bank is a lender of last resort for the Government, which then has no accounting limits to spending.

\section{Real Assets}
\subsection{Consumption Goods}
Consumption Goods represents all the consumption (goods and services) of Households and are homogeneus and non-durable.

\begin{enh}[Foundational Economy -- 3]
	The exact definition of Foundational good (and service) is not easy to give. An intuition can be provided by the Foundational Economy approach \parencite{arcidiacono2018}: \begin{quote}
		The sphere of the foundational was then demarcated by three criteria: these goods and services were necessary to everyday life; were consumed daily by all citizens regardless of income; and were distributed according to population through branches and networks. They were partly non-market, generally sheltered and one way or another politically franchised.
	\end{quote}

	Operationally, we can imagine the essential goods in a future model as the ones included in the basket used by national statistics offices to determine the poverty line. In this sense, it is a set of goods which continuously mutate to adapt to new life needs.
\end{enh}

\begin{enh}[Housing]
	Among essential goods, one should require ad hoc modelling: houses. Houses are special for three reasons.

	First, they are very heterogeneous in prices and quality, and both are strongly related to the position. In other words, including houses requires (quite always) making the model spatially explicit.

	Second, the expenses for housing, in form of rent or mortgage, account for a significant part of monthly consumptions for poor individuals (up to one-half).

	Third, real estate properties are an important form of rent extraction and an important tool of investment, and so another important channel of redistribution.
\end{enh}

\begin{enh}[Diversified Goods]
	A subsequent version of the model can include different (abstract) goods to be produced and consumed. This will create two different innovation processes (better technology for existing goods, or technology for new goods) and will account for the empirical fact that the higher the income more diversified the consumptions are \parencite[cfr.][§2]{didomenico2022}.
\end{enh}

\subsection{Capital Goods}
Capital goods are characterized by their productivity $\beta$ and the minimum skill level required to operate them $\sigma$. The production happens according to a Leontieff-like function in which one worker with at least $\sigma$ skill can operate a single capital good \hl{ getting $\beta$ goods as output}\footnote{It is possible that, to balance the model, the output needs to be $\beta$ consumption goods or $b\beta$ capital goods.}. Each period they have a fixed probability to break and disappear from the model, equal to $\delta = \langle N \rangle^{-1}$, where $\langle N \rangle$ is the expected life of the machinery.

\section{Financial Assets}

\subsection{Deposits}
Deposits represent liquidity for Households and Firms and are not interest-bearing. Bank satisfies any transaction as long as the balance of the account remains positive.

\subsection{Bank Bonds}
Bank Bonds are sold and bought at their nominal value and do not expire. Bank satisfies every transaction, as long as the accounts remain positive. Households can buy or sell Bank Bonds only at the beginning of each period. At the end of each period, interests are paid, according to the rate fixed by the Bank at the beginning of the period.

\subsection{Loans}
Loans are issued by the Bank to a specific Firm. They have a fixed duration during which an equal share of capital is repaid plus the interest on the remaining debt. Interests are fixed by the Bank at a different value for each Firm at the time of emission.

\subsection{Government Bonds}
Government Bonds are sold and bought at their nominal value and do not expire. Central Bank satisfies every transaction. Bank can buy Government Bonds only at the beginning of each period. At the end of each period, interests are paid only to the Bank, according to the rate fixed by the Central Bank at the time beginning of the period.

\subsection{Reserves}
Reserves represent liquidity for Bank and Government and are not interest-bearing. Central Bank satisfies any transaction as long as the balance of the account remains positive.

\section{Dynamics}
\subsection{Consumption Goods market}
Each Household sees $\chi_C$ Consumption Firms $\chi_C$ times, each time it buys $\frac{\mathbb{E}(G) - G^G}{\chi_C}$ at the lowest possible price, untill it matches the desired consumption or ends liquidity.

\subsection{Capital Goods market}
Each Consumption Firm sees $\chi_K$ Capital Firms $\chi_K$ times (including the last one from which it has bought), each time it buys $\frac{\Delta B}{\chi_C}$ machineries chosing the one with the lowest value of $\beta p_C - \langle w_\sigma \rangle - \frac{p_K}{\langle N \rangle}$, untill it matches the desired quantity or ends liquidity.

\subsection{Labour market}
Households are employed by a Firm untill they are fired, they chose to exit the job market or they accept an offer from another Firm.

Firm can fire workers only when profits turn negative. In that case, Firms fire workes untill the number of workers reaches $u^* K$. Additionally, in the case the Firm is not able to pay a worker, it is fired. In both cases, Firms start to fire from those with the lower skills.

Firms match each employed worker with the most productive machinery they can use, starting from those with higher skills. For this purpose a reasercher is a worker that is assigned to a machinery with productivity $\sigma^*$.
Then the current potential output (as the sum of the productivity of the used machineries) is computed and vacancies are filled starting from the most productive unused machineries. All research vacancies are filled if possible.
For each vacancy, the Firm sees $\chi_L$ workers which have the required skills and earn less than the average salary in the model for the skill level. The Firm employs the one with the higher skills offering the average salary in the model for the skill level.

Each vacancies is filled once and new vacancies are not filled.

\begin{enh}[Collective bargain]

\end{enh}

\subsection{Retirement and inheritance}

\subsection{Non Performing Loans}

\subsection{Bankrupt}

\subsection{Innovation}

\section{Model steps}
\begin{steps}
	\item
\end{steps}

\section{Equations}

\section{Parameters}

\printbibliography

\end{document}
